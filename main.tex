\documentclass[12pt]{article}
\usepackage[utf8]{inputenc}
\usepackage[T1]{fontenc}
\usepackage[english]{babel}
\usepackage{amsmath, amssymb, amsthm}
\usepackage{geometry}
\usepackage{titling}
\usepackage{fancyhdr}
\usepackage{lipsum}
\usepackage{parskip}
\usepackage{forest}
\usepackage{tikz}
\usepackage{stmaryrd}
\usepackage{listings}
\usepackage{graphicx}
\usepackage{float}
\usepackage{alphalph}
\usepackage{cancel}
\usepackage{textgreek}
\usepackage{titlesec}
\usepackage{dsfont}
\usepackage{caption}
\usepackage{listings}

\geometry{top=4cm, bottom=4cm, left=4cm, right=4cm}
\pagestyle{fancy}
\fancyhf{}
\rhead{Pierre Pili $\cdot$ Marie Gardie $\cdot$ Isée Biglietti}
\lhead{Econometrics 3}
\cfoot{\thepage}
\setlength{\headheight}{14.49998pt}
\addtolength{\topmargin}{-2.49998pt}

\titleformat{\section}{\small\bfseries}{\thesection}{1em}{}
\renewcommand{\thesubsection}{\arabic{section}.\arabic{subsection}}
\renewcommand{\thesubsubsection}{\arabic{section}.\arabic{subsection}.\alph{subsubsection}}

\title{Problem Set 7}
\author{PILI Pierre $\cdot$ GARDIE Marie $\cdot$ BIGLIETTI Isée}
\date{\today}

% Redéfinir le format de numérotation des sous-sections
\titleformat{\subsection}
  {\normalfont\small\bfseries}{\alph{subsection})}{1em}{}
  
\begin{document}
\maketitle
\renewcommand{\thesubsection}{\alph{subsection}}

\subsection{Regress the variable $survived$ on $female$. Report and interpret the estimated (marginal) “effect” of being female.}

% Table created by stargazer v.5.2.3 by Marek Hlavac, Social Policy Institute. E-mail: marek.hlavac at gmail.com
% Date and time: Mer, avr 24, 2024 - 15:56:03
\begin{table}[!htbp] \centering 
  \caption{OLS} 
  \label{olsa} 
\begin{tabular}{@{\extracolsep{5pt}}lc} 
\\[-1.8ex]\hline 
\hline \\[-1.8ex] 
 & \multicolumn{1}{c}{\textit{Dependent variable:}} \\ 
\cline{2-2} 
\\[-1.8ex] & survived \\ 
\hline \\[-1.8ex] 
 female & 0.536$^{***}$ \\ 
  & (0.024) \\ 
  & \\ 
 Constant & 0.191$^{***}$ \\ 
  & (0.014) \\ 
  & \\ 
\hline \\[-1.8ex] 
Observations & 1,309 \\ 
R$^{2}$ & 0.280 \\ 
Adjusted R$^{2}$ & 0.279 \\ 
Residual Std. Error & 0.413 (df = 1307) \\ 
F Statistic & 507.059$^{***}$ (df = 1; 1307) \\ 
\hline 
\hline \\[-1.8ex] 
\textit{Note:}  & \multicolumn{1}{r}{$^{*}$p$<$0.1; $^{**}$p$<$0.05; $^{***}$p$<$0.01} \\ 
\end{tabular} 
\end{table} 

The dependent variable $survived$ is a binary variable. We are thus in the binary outcome framework. In this question we perform an OLS regression of the $survived$
variable on the $female$ variable. The average survival rate was $0.191$ (see Table \ref{olsa}) while a woman would survive with a probability $0.191 + 0.536 = 0.727$.
Being a female increases your probability of survival by $0.536$. As the $female$ variable is binary, those numbers can be interpreted as probabilities. The result is very significant, being a female on board dramatically increased your chances of survival.
\subsection{Construct a 95\% confidence interval for the estimated (marginal) effect.}
Using the robust estimated variance we find a confidence interval at the 95\% level for the estimated marignal effect equal to $[0.488, 0.585]$.
\subsection{Repeat (a) using the probit. What do you find ? Are you surprised ?}

% Table created by stargazer v.5.2.3 by Marek Hlavac, Social Policy Institute. E-mail: marek.hlavac at gmail.com
% Date and time: Mer, avr 24, 2024 - 15:56:03
\begin{table}[!htbp] \centering 
  \caption{Probit Regression} 
  \label{prbtc} 
\begin{tabular}{@{\extracolsep{5pt}}lc} 
\\[-1.8ex]\hline 
\hline \\[-1.8ex] 
 & \multicolumn{1}{c}{\textit{Dependent variable:}} \\ 
\cline{2-2} 
\\[-1.8ex] & survived \\ 
\hline \\[-1.8ex] 
 female & 1.479$^{***}$ \\ 
  & (0.080) \\ 
  & \\ 
 Constant & $-$0.874$^{***}$ \\ 
  & (0.050) \\ 
  & \\ 
\hline \\[-1.8ex] 
Observations & 1,309 \\ 
Log Likelihood & $-$684.052 \\ 
Akaike Inf. Crit. & 1,372.103 \\ 
\hline 
\hline \\[-1.8ex] 
\textit{Note:}  & \multicolumn{1}{r}{$^{*}$p$<$0.1; $^{**}$p$<$0.05; $^{***}$p$<$0.01} \\ 
\end{tabular} 
\end{table} 

The coefficient of interest in the probit regression is $\beta = 1.479$ (see Table \ref{prbtc}) which means that the marginal effect is positive which was expected, this is not the marginal effect however, to compute partial effects one must compute $\beta \phi(\beta) = 0.198$ for the marginal effect when $female = 1$ and $\beta \phi(0) = 0.590$ when $female = 0$. The average marginal effect is somewhere in between. Using the margins package, I find a marginal effect of $0.434$. I do not find those results very surprising, it seems that the average marginal effect is lower than the one OLS came up with however. 
\stepcounter{subsection}
\subsection{Continuing with the probit model from (c), add a numeric copy of the variable pclass to the “regression”. Interpret your results.}
As in the previous question, we find a significant positive impact of being a female on board and a significant negative impact of the $pclass$ variable (see Table \ref{prbte}). Using the library margins, I find a marginal effect of $0.406$ for the female variable and $-0.133$ for the $pclass$ variable. The intuition is clear, the $pclass$ goes from $1$ to $3$ from the most expensive seat to the cheapest. Being in first class increased your chances of surival by a significant amount, but still less than being a female. Leonardo Dicaprio increased both coefficients in absolute value even though their was enough space on this little piece of wood...

% Table created by stargazer v.5.2.3 by Marek Hlavac, Social Policy Institute. E-mail: marek.hlavac at gmail.com
% Date and time: Mer, avr 24, 2024 - 15:56:04
\begin{table}[!htbp] \centering 
  \caption{Probit Regression} 
  \label{prbte} 
\begin{tabular}{@{\extracolsep{5pt}}lc} 
\\[-1.8ex]\hline 
\hline \\[-1.8ex] 
 & \multicolumn{1}{c}{\textit{Dependent variable:}} \\ 
\cline{2-2} 
\\[-1.8ex] & survived \\ 
\hline \\[-1.8ex] 
 female & 1.503$^{***}$ \\ 
  & (0.083) \\ 
  & \\ 
 pclass & $-$0.494$^{***}$ \\ 
  & (0.048) \\ 
  & \\ 
 Constant & 0.245$^{**}$ \\ 
  & (0.117) \\ 
  & \\ 
\hline \\[-1.8ex] 
Observations & 1,309 \\ 
Log Likelihood & $-$629.609 \\ 
Akaike Inf. Crit. & 1,265.218 \\ 
\hline 
\hline \\[-1.8ex] 
\textit{Note:}  & \multicolumn{1}{r}{$^{*}$p$<$0.1; $^{**}$p$<$0.05; $^{***}$p$<$0.01} \\ 
\end{tabular} 
\end{table} 

\subsection{Instead of adding a numeric copy of the variable pclass to the regression in (c), add the variable pclass to it (pclass is coded as a “factor”). Interpret your results.}

% Table created by stargazer v.5.2.3 by Marek Hlavac, Social Policy Institute. E-mail: marek.hlavac at gmail.com
% Date and time: Mar, avr 23, 2024 - 11:49:31
\begin{table}[!htbp] \centering 
  \caption{Probit Regression} 
  \label{prbtf} 
\begin{tabular}{@{\extracolsep{5pt}}lc} 
\\[-1.8ex]\hline 
\hline \\[-1.8ex] 
 & \multicolumn{1}{c}{\textit{Dependent variable:}} \\ 
\cline{2-2} 
\\[-1.8ex] & survived \\ 
\hline \\[-1.8ex] 
 female & 1.503$^{***}$ \\ 
  & (0.083) \\ 
  & \\ 
 pclass2 & $-$0.537$^{***}$ \\ 
  & (0.115) \\ 
  & \\ 
 pclass3 & $-$0.994$^{***}$ \\ 
  & (0.097) \\ 
  & \\ 
 Constant & $-$0.236$^{***}$ \\ 
  & (0.083) \\ 
  & \\ 
\hline \\[-1.8ex] 
Observations & 1,309 \\ 
Log Likelihood & $-$629.527 \\ 
Akaike Inf. Crit. & 1,267.055 \\ 
\hline 
\hline \\[-1.8ex] 
\textit{Note:}  & \multicolumn{1}{r}{$^{*}$p$<$0.1; $^{**}$p$<$0.05; $^{***}$p$<$0.01} \\ 
\end{tabular} 
\end{table} 

This time the models estimates two coefficients for two dummy variables, $pclass2$ which is equal to $1$ when the passager is in second class, and similarly for $pclass3$. This regression adds information to the previous one as only one coefficient was used to describe an incrementation in the pclass variable. We see that, as expected, the cheaper the seat you bought, the less chances of survival you had. Using the library margins, I find a marginal effect of $0.406$ for the female variable, $-0.169$ for the $pclass2$ variable and $-0.295$. It means that the impact from moving from first class to the third is about twice as bad as moving from first class to the second. Put otherwise, the effect of a one unit incrementation of the pclass variable seems constant.

\subsection{The model in (e) constitutes a restricted version of the model in (f). What is the restriction? Test it using a LR test. What do you find?}
The model estimated in question (f) writes
$$survived = \phi(\alpha + \beta_1 female + \beta_2 pclass2 + \beta_3 pclass 3)$$
while the model the model in question (e) imposes the restriction $\beta_3 = 2 \beta_2$.
\stepcounter{subsection}
\subsection{Starting with your model specification in (e), add the variable $fare$. Use the Wald test and the LR test to test whether the coefficient on $fare$ is equal to zero. What do you find?}
\subsection{Starting with your model specification in (e), add the variable age. Compute the partial effect of age for a female passenger in 1st class with “average” age.}

% Table created by stargazer v.5.2.3 by Marek Hlavac, Social Policy Institute. E-mail: marek.hlavac at gmail.com
% Date and time: Mer, avr 24, 2024 - 15:56:06
\begin{table}[!htbp] \centering 
  \caption{Probit Regression} 
  \label{prbtj} 
\begin{tabular}{@{\extracolsep{5pt}}lc} 
\\[-1.8ex]\hline 
\hline \\[-1.8ex] 
 & \multicolumn{1}{c}{\textit{Dependent variable:}} \\ 
\cline{2-2} 
\\[-1.8ex] & survived \\ 
\hline \\[-1.8ex] 
 female & 1.484$^{***}$ \\ 
  & (0.094) \\ 
  & \\ 
 pclass & $-$0.641$^{***}$ \\ 
  & (0.062) \\ 
  & \\ 
 age & $-$0.019$^{***}$ \\ 
  & (0.004) \\ 
  & \\ 
 Constant & 1.160$^{***}$ \\ 
  & (0.215) \\ 
  & \\ 
\hline \\[-1.8ex] 
Observations & 1,046 \\ 
Log Likelihood & $-$492.797 \\ 
Akaike Inf. Crit. & 993.594 \\ 
\hline 
\hline \\[-1.8ex] 
\textit{Note:}  & \multicolumn{1}{r}{$^{*}$p$<$0.1; $^{**}$p$<$0.05; $^{***}$p$<$0.01} \\ 
\end{tabular} 
\end{table} 

\subsection{Using your model specification in (j), compute the average partial effect of age. What do you find?}
\subsection{Using your model specification in (j), plot the sorted marginal effect of age. What do you find ?}
\end{document}



