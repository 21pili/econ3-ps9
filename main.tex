\documentclass[12pt]{article}
\usepackage[utf8]{inputenc}
\usepackage[T1]{fontenc}
\usepackage[english]{babel}
\usepackage{amsmath, amssymb, amsthm}
\usepackage{geometry}
\usepackage{titling}
\usepackage{fancyhdr}
\usepackage{lipsum}
\usepackage{parskip}
\usepackage{forest}
\usepackage{tikz}
\usepackage{stmaryrd}
\usepackage{listings}
\usepackage{graphicx}
\usepackage{float}
\usepackage{alphalph}
\usepackage{cancel}
\usepackage{textgreek}
\usepackage{titlesec}
\usepackage{dsfont}
\usepackage{caption}
\usepackage{listings}

\geometry{top=4cm, bottom=4cm, left=4cm, right=4cm}
\pagestyle{fancy}
\fancyhf{}
\rhead{Pierre Pili $\cdot$ Marie Gardie $\cdot$ Isée Biglietti}
\lhead{Econometrics 3}
\cfoot{\thepage}
\setlength{\headheight}{14.49998pt}
\addtolength{\topmargin}{-2.49998pt}

\titleformat{\section}{\small\bfseries}{\thesection}{1em}{}
\renewcommand{\thesubsection}{\arabic{section}.\arabic{subsection}}
\renewcommand{\thesubsubsection}{\arabic{section}.\arabic{subsection}.\alph{subsubsection}}

\title{Problem Set 7}
\author{PILI Pierre $\cdot$ GARDIE Marie $\cdot$ BIGLIETTI Isée}
\date{\today}

% Redéfinir le format de numérotation des sous-sections
\titleformat{\subsection}
  {\normalfont\small\bfseries}{\alph{subsection})}{1em}{}
  
\begin{document}
\maketitle
\renewcommand{\thesubsection}{\alph{subsection}}

\subsection{Regress the variable $survived$ on $female$. Report and interpret the estimated (marginal) “effect” of being female.}

% Table created by stargazer v.5.2.3 by Marek Hlavac, Social Policy Institute. E-mail: marek.hlavac at gmail.com
% Date and time: Mer, avr 24, 2024 - 15:56:03
\begin{table}[!htbp] \centering 
  \caption{OLS} 
  \label{olsa} 
\begin{tabular}{@{\extracolsep{5pt}}lc} 
\\[-1.8ex]\hline 
\hline \\[-1.8ex] 
 & \multicolumn{1}{c}{\textit{Dependent variable:}} \\ 
\cline{2-2} 
\\[-1.8ex] & survived \\ 
\hline \\[-1.8ex] 
 female & 0.536$^{***}$ \\ 
  & (0.024) \\ 
  & \\ 
 Constant & 0.191$^{***}$ \\ 
  & (0.014) \\ 
  & \\ 
\hline \\[-1.8ex] 
Observations & 1,309 \\ 
R$^{2}$ & 0.280 \\ 
Adjusted R$^{2}$ & 0.279 \\ 
Residual Std. Error & 0.413 (df = 1307) \\ 
F Statistic & 507.059$^{***}$ (df = 1; 1307) \\ 
\hline 
\hline \\[-1.8ex] 
\textit{Note:}  & \multicolumn{1}{r}{$^{*}$p$<$0.1; $^{**}$p$<$0.05; $^{***}$p$<$0.01} \\ 
\end{tabular} 
\end{table} 

The dependent variable $survived$ is a binary variable. We are thus in the binary outcome framework. In this question we perform an OLS regression of the $survived$
variable on the $female$ variable. The average survival rate was $0.191$ (see Table \ref{olsa}) while a woman would survive with a probability $0.191 + 0.536 = 0.727$.
Being a female increases your probability of survival by $0.536$. As the $female$ variable is binary, those numbers can be interpreted as probabilities. The result is very significant, being a female on board dramatically increased your chances of survival.
\subsection{Construct a 95\% confidence interval for the estimated (marginal) effect.}
Using the robust estimated variance we find a confidence interval at the 95\% level for the estimated marignal effect equal to $[0.488, 0.585]$.
\subsection{Repeat (a) using the probit. What do you find ? Are you surprised ?}

% Table created by stargazer v.5.2.3 by Marek Hlavac, Social Policy Institute. E-mail: marek.hlavac at gmail.com
% Date and time: Mer, avr 24, 2024 - 15:56:03
\begin{table}[!htbp] \centering 
  \caption{Probit Regression} 
  \label{prbtc} 
\begin{tabular}{@{\extracolsep{5pt}}lc} 
\\[-1.8ex]\hline 
\hline \\[-1.8ex] 
 & \multicolumn{1}{c}{\textit{Dependent variable:}} \\ 
\cline{2-2} 
\\[-1.8ex] & survived \\ 
\hline \\[-1.8ex] 
 female & 1.479$^{***}$ \\ 
  & (0.080) \\ 
  & \\ 
 Constant & $-$0.874$^{***}$ \\ 
  & (0.050) \\ 
  & \\ 
\hline \\[-1.8ex] 
Observations & 1,309 \\ 
Log Likelihood & $-$684.052 \\ 
Akaike Inf. Crit. & 1,372.103 \\ 
\hline 
\hline \\[-1.8ex] 
\textit{Note:}  & \multicolumn{1}{r}{$^{*}$p$<$0.1; $^{**}$p$<$0.05; $^{***}$p$<$0.01} \\ 
\end{tabular} 
\end{table} 

\stepcounter{subsection}
\subsection{Continuing with the probit model from (c), add a numeric copy of the variable pclass to the “regression”. Interpret your results.}
\subsection{Instead of adding a numeric copy of the variable pclass to the regression in (c), add the variable pclass to it (pclass is coded as a “factor”). Interpret your results.}
\subsection{The model in (e) constitutes a restricted version of the model in (f). What is the restriction? Test it using a LR test. What do you find?}
\stepcounter{subsection}
\subsection{Starting with your model specification in (e), add the variable $fare$. Use the Wald test and the LR test to test whether the coefficient on $fare$ is equal to zero. What do you find?}
\subsection{Starting with your model specification in (e), add the variable age. Compute the partial effect of age for a female passenger in 1st class with “average” age.}
\subsection{Using your model specification in (j), compute the average partial effect of age. What do you find?}
\subsection{Using your model specification in (j), plot the sorted marginal effect of age. What do you find ?}
\end{document}



